\documentclass[10pt, a4paper]{article}
\usepackage{fullpage}
\usepackage{amsmath, amssymb, amsthm, amsfonts, physics}
\usepackage{array}
\usepackage{enumerate}
\usepackage{pbox}
\theoremstyle{plain}
\newtheorem{thrm}{Theorem}[section]
\theoremstyle{definition}
\newtheorem{defn}[thrm]{Definition}
\newtheorem{exmp}[thrm]{Example}
\DeclareMathOperator{\Ima}{Im}
\DeclareMathOperator{\id}{id}
\DeclareMathOperator{\arcsinh}{arcsinh}
\DeclareMathOperator{\gal}{Gal}
\DeclareMathOperator{\disc}{disc}
\DeclareMathOperator{\Nm}{Nm}
\DeclareMathOperator{\Div}{Div}
\DeclareMathOperator{\coker}{Coker}
\DeclareMathOperator{\sym}{Sym}
\DeclareMathOperator{\Hom}{Hom}
\DeclareMathOperator{\Tra}{Tr}
\DeclareMathOperator{\Ad}{Ad}
\DeclareMathOperator{\ad}{ad} 
\DeclareMathOperator{\GL}{GL} 
\DeclareMathOperator{\proj}{proj} 
\DeclareMathOperator{\Jac}{Jac} 
\DeclareMathOperator{\LB}{LB} 
\DeclareMathOperator{\gl}{\mathfrak{gl}}
\DeclareMathOperator{\SO}{SO} 
\DeclareMathOperator{\SL}{SL} 
\DeclareMathOperator{\U}{U} 
\DeclareMathOperator{\SU}{SU} 
\DeclareMathOperator{\End}{End} 
\DeclareMathOperator{\Deck}{Deck} 
\renewcommand{\Vert}{\text{Vert}} 
\renewcommand{\div}{\text{div}} 
\DeclareMathOperator{\Aut}{Aut} 
\DeclareMathOperator{\Der}{\mathfrak{Der}}
\DeclareMathOperator{\Int}{\mathfrak{Int}} 
\DeclareMathOperator{\Inn}{Int} 
\DeclareMathOperator{\Isom}{Isom} 
\DeclareMathOperator{\Ric}{Ric} 
\DeclareMathOperator{\Pic}{Pic} 
\DeclareMathOperator{\mult}{mult} 
\DeclareMathOperator{\spn}{span}
\DeclareMathOperator{\sgn}{sgn}
\DeclareMathOperator{\Diff}{Diff}
\DeclareMathOperator{\On}{O}
\DeclareMathOperator{\ord}{ord}
\DeclareMathOperator{\PDiv}{PDiv}
\DeclareMathOperator{\Lie}{Lie}
\DeclareMathOperator{\Homeo}{Homeo}
\renewcommand{\ip}[2]{\langle #1, #2 \rangle}
\allowdisplaybreaks
\title{British Isles Graduate Workshop: Higgs Bundles \\
Talk 1: Riemann surfaces}
\author{Marielle Ong}
\begin{document}
\maketitle
\section{Definition of a Riemann surface}
\begin{defn} A \emph{Riemann surface} $X$ is a complex manifold of dimension 1. 
\end{defn} 
Since $\mathbb{C}$ can be identified with $\mathbb{R}^2$, we can view $X$ as a real, orientable manifold of dimension 2.  
\begin{exmp} The complex projective line $\mathbb{P}^1(\mathbb{C})$ 
is the set of one-dimensional subspaces of $\mathbb{C}^2$. That is, if $(x, y)$ is a non-zero vector in $\mathbb{C}^2$, then its span, denoted by $[x : y],$ is a point in $\mathbb{P}^1(\mathbb{C}).$ It is a Riemann surface with charts $U_i$ and coordinate maps $\phi_i$ given by 
\begin{align*} 
U_1 &= \{[x: y] : x \neq 0 \},  \quad \phi_1([x: y]) = y/x, \\
U_2 &= \{[x: y] : y \neq 0 \},  \quad \phi_2([x: y]) = x/y. 
\end{align*}
These charts are compatible because the composition
$$\phi_2 \circ \phi_1^{-1}: \phi_1(U_1 \cap U_2) \rightarrow \phi_2(U_1 \cap U_2),$$
which is given by $\phi_2 \circ \phi_1^{-1}(z) = 1/z$, is biholomorphic on $U_1 \cap U_2 = \mathbb{C}^*$. 
\end{exmp}
\begin{exmp} Let $\omega_1, \omega_2 \in \mathbb{C}$ be linearly independent over $\mathbb{R}$. Define the lattice 
$$\Gamma(\omega_1, \omega_2) = \mathbb{Z}\omega_1  + \mathbb{Z}\omega_2,$$
which is a subgroup of the additive group $\mathbb{C}$. We call the quotient $\mathbb{C}/\Gamma$ the complex torus. It is a Riemann surface equipped with the quotient topology via the projection map $\pi: \mathbb{C} \rightarrow X$.
\end{exmp} 
More detailed examples can be found in Chapter 1 of Rick Miranda's book "Algebraic curves and Riemann surfaces". For notation, denote the set of holomorphic functions over an open subset $U \subseteq X$ by $\mathcal{O}_X(U)$ and similarly, the set of meromorphic functions over an open subset $U \subseteq X$ as $\mathcal{M}_X(U)$. 
\section{Different perspectives on Riemann surfaces}
Riemann surfaces may be described in four different settings: topological (Betti), smooth (De Rham), holomorphic (Dolbeault) and algebraic. 
\begin{center}
\begin{tabular}{|c|c|c|c|}
\hline \textbf{Topological} & \textbf{Smooth} & \textbf{Dolbeault} & \textbf{Algebraic} \\ 
\hline 
\pbox{20cm}{\vspace{0.1cm}
\textbf{Theorem:} Every closed \\ orientable  surface is \\ homeomorphic to \\ a sphere with $g$ handles, \\
denoted by $\Sigma_g$.} & \pbox{20cm}{\vspace{0.1cm} \textbf{Theorem:} Let $M$ be a smooth, \\ orientable manifold \\with dimension 2. \\
Then, $M$ is diffeomorphic \\ to $\Sigma_g$.} & GAGA & GAGA \\ 
\hline 
\pbox{20cm}{\vspace{0.1cm}  $\pi_1(\Sigma_g) =$ \\ $\{a_i, b_i : \prod^g_{i = 1} a_i b_i a_i^{-1} b_i^{-1} = e\}$}  & \pbox{20cm}{\vspace{0.1cm}  \textbf{Theorem (de Rham):} \\  $H^k_{\text{sing}}(X, \mathbb{R}) \cong H^k_{\text{DR}}(X)}$ & \pbox{20cm}{\vspace{0.1cm}   holomorphic objects \\ holomorphic maps \\ Riemann surfaces \\ complex tori} & \pbox{20cm}{\vspace{0.1cm}  algebraic varieties \\ regular morphisms \\ Projective varieties \\elliptic curves}  \\ 
\hline \vspace{0.1cm}  \pbox{20cm}{$H_k( \Sigma_g) = \begin{cases} \mathbb{Z} & k = 0, 2 \\
\mathbb{Z}^{2g} & k = 1 \end{cases}}$ &  &  & • \\ 
\hline 
\end{tabular} 
\end{center}
\section{Line bundles}
\begin{defn} A \emph{holomorphic line bundle} on $X$ is a complex manifold $L$ and a holomorphic surjective map $\pi: L \rightarrow X$ such that 
\begin{enumerate}[(i)]
\item For every $p \in X$, $L_p = \pi^{-1}(p) \cong \mathbb{C}.$
\item For every $p \in X$, there exists a neighbourhood $U$ containing $p$ and a vector space isomorphism $\phi_U: \pi^{-1}(U) \rightarrow U \times \mathbb{C}$. We call $\phi_U$ a local trivialisation on $U$. 
\item the local trivialisations relate to each other via transition maps $t_{ij}$, which satisfy cocyle relations.
\end{enumerate}
\end{defn}
Under the tensor product $\otimes$, the isomorphism classes of line bundles form a group $\Pic(X)$. Here, the inverse of a line bundle $L$ is given by its dual $L^*$ and the trivial bundle $O = \End(L) = L \otimes L^*$ is the identity. 
\\\\
Let $L$ be a line bundle with a covering $\{U_i\}_{i \in I}$ of $X$. 
\begin{defn} A \emph{holomorphic (resp. meromorphic) section} $s$ is a collection of sections $\{s_i\}_{i \in I|}$ over $U_i$ such that $t_{ij} s_i = s_j$ for $s_i \in \mathcal{O}_X(U_i)$ (resp. $\mathcal{M}_X(U_i)$)
and $t_{ij} \in \mathcal{O}_X(U_i \cap U_j)$ (resp. $\mathcal{O}^*_X(U_i \cap U_j)$. 
\end{defn} 
We  denote the sheaf of holomorphic sections as $\mathcal{O}_L$. 
\begin{defn} A sheaf is \emph{invertible} if for all open sets $U_i$ in the cover $\{U_i\}_{i \in I}$, the group $\mathcal{F}(U_i)$ is a free $\mathcal{O}_X(U_i)$-module. 
\end{defn}
This condition is what we call ``locally free of rank 1". If it is of rank $n$, then $\mathcal{F}(U_i)$ is isomorphic to a direct sum of $n$ copies of $\mathcal{O}_X(U_i)$-modules. 
\begin{thrm} There exists a canonical isomorphism 
$$\Pic(X) \cong \left\{\text{group of invertible sheaves on $X$ under } \otimes \right\} \cong H^1(X, \mathcal{O}^*),$$
where $\mathcal{O}^*$ is the sheaf of nowhere vanishing holomorphic functions.
\end{thrm} 
\begin{defn} The \emph{canonical bundle} $K$ is the determinant if the cotangent bundle. For the case of Riemann surfaces, $K = \Omega^1_X$, the bundle of holomorphic $1$-forms. 
\end{defn} 
\begin{thrm}[Riemann-Roch]
If $L$ is a line bundle over a Riemann surface $X$ with genus $g$, then
$$\dim H^0(X, \mathcal{O}_X) - \dim H^0(X, \mathcal{O}_{L^{-1} \otimes K} = \deg(L)+1-g.$$
\end{thrm} 
Letting $L$ be the trivial bundle, one can show that $\dim H^0(X, \mathcal{O}_K) = g$. Therefore, $g$ is the dimension of the space of holomorphic 1-forms. 
\begin{thrm}[Serre Duality]
$H^1(X, \mathcal{O}_L) \cong H^0(X, \mathcal{O}_{L^{-1} \otimes K})^*$. 
\end{thrm} 
The Picard group has some interesting geometric structures and to see this, we need to realise line bundles via divisors. 
\section{Divisors}
\begin{defn} A \emph{divisor} is a function $D: X \rightarrow \mathbb{Z}$, denoted by 
$$D = \sum_{p \in X} n_p \cdot p$$
with $n_p \in \mathbb{Z}$. Furthermore, the \emph{degree} of a divisor is given by
$$\deg(D) = \sum_{p\in X} n_p.$$
\end{defn}
The divisors form a group, which we will denote by $\Div X$. Let $\Div_0 X$ be the group of degree 0 divisors. 
\begin{defn} A divisor is \emph{principal} if it is of the form $\div(f) = \sum_{p \in X} \ord_p(f) \cdot p$. The group of principal divisors is $\PDiv X$. 
\end{defn}
We say that two divisors $D_1$ and $D_2$ are linearly equivalent or $D_1 \sim D_2$ if and only if $D_1 - D_2 \in \PDiv X$. Through this linear equivalence, we can form the group
$$\frac{\Div X}{\PDiv X}.$$
We say that a divisor $D \geq 0$ if $n_p \geq 0$ for all $p \in X$. From this, we can associate a vector space of meromorphic functions to a divisor. 
\begin{defn} 
The \emph{vector space of meromorphic functions associated to a divisor} is given by 
$$L(D) = \{f \in \mathcal{M}(X) : \div(f) \geq -D\}.$$
\end{defn} 
Let $s$ be a meromorphic section. Define 
\begin{align*}
\ord_p(s) &= \ord_p(s_i), \quad p \in U_i \\
\div(s) &= \sum_{p \in X} \ord_p(s) \cdot p \\
\deg(L) = \sum_{p \in X} \ord_p(s).
\end{align*}
So from a given section of a line bundle, we can obtain a divisor. It turns out that the converse is true: that divisors give rise to line bundles. One first needs to be able to associate a meromorphic function to a divisor. This is called the \emph{Weierstrass problem}, which states: given a divisor $D$, is there a meromorphic function $f$ such that $\div(f) = D$? This problem is always solvable over non-compact sets. For the situation over compact sets, one requires content in the \emph{Abel-Jacobi theorem}, which we will touch on later. By transitioning between meromorphic function on different neighbourhoods, we obtain a transition function, which completely determines a line bundle. As such, the line bundle-divisor correspondence may be stated as follows. 
\begin{thrm}
There exists canonical isomorphisms 
$$\frac{\Div X}{\PDiv X} \cong \Pic(X) \quad \frac{\Div_0 X}{\PDiv X} \cong \Pic_0(X)$$
where $\Pic_0(X)$ is the line bundles with degree 0. 
\end{thrm}
\section{The Jacobian}
Let $H^0(X, \Omega^1_X)$ to be the space of holomorphic 1-forms. Define 
\begin{align*} 
\lambda_c: H^0(X, \Omega^1_X) &\rightarrow \mathbb{C} \\
\omega \mapsto \int_c d\omega.
\end{align*}
By Stoke's theorem, $\lambda_c$ only depends on the homology class $[c] \in H_1(X, \mathbb{Z})$. We obtain a map 
$$\lambda: H_1(X, \mathbb{Z}) \rightarrow H^0(X, \Omega^1_X)^*.$$
Let $\Lambda = \im(\lambda)$. 
\begin{defn} The \emph{Jacobian} of $X$ is given by 
$$\Jac(X) = \frac{H^0(X, \Omega^1_X)^*}{\Lambda}.$$
\end{defn} 
Since $\dim H^0(X, \Omega^1_X)^* = g$ (as a consequence of Riemann Roch), we can identify $ H^0(X, \Omega^1_X)^*$ with $\mathbb{C}^g$ and so, $\Jac(X) \cong \mathbb{C}^g/\Lamgda$, the $g$-dimensional torus. 
\begin{defn} Fix a base point $p_0 \in X$. For all $p \in X$, define $A: X \rightarrow \Jac(X)$ as the \emph{Abel-Jacobi map} in which 
$$p \mapsto \int_{\gamma_p} \omega,$$
where $\gamma_p$ is a path from $p_0$ to $p$. \end{defn}
Extend the Abel-Jacobi map linearly to 
$$A(D) = \sum_{p \in X} n_p A(p)$$
and obtain the map $A: \Div X \rightarrow \Jac X$ and its restriction to degree 0 divisors, $A_0: \Div_0 X \rightarrow \Jac X$. 
\begin{thrm}[Abel]
Let $D \in Div_0 X$. If $A_0(D) = 0$, then $D \in \PDiv X$.  
\end{thrm} 
\begin{thrm}[Jacobi] The map $A_0: \Div_0 X \rightarrow \Jac X$ is surjective.
\end{thrm}
With these theorems combined, we have that
$$\Pic_0 X \cong \frac{\Div_0 X}{\PDiv X} \cong \Jac X.$$
Using the splitting of the exact sequence 
$$0 \rightarrow \Pic_0 X \rightarrow \Pic X \rightarrow \mathbb{Z} \rightarrow 0,$$
we have the isomorphism
$$\Pic X \cong \Pic_0 X \times \mathbb{Z} \cong \Jac X \times \mathbb{Z}.$$
So the upshot of this is that line bundles are parametrized by $\Jac X \times \mathbb{Z}$. 
\section{Hodge theory}
Let $X$ be a compact complex manifold and $TX$ be its underlying real tangent bundle. The complex structure of $X$ induces an almost complex structure $J$ on $TX$. Here, $J$ is a $\mathbb{R}$-linear endomorphism,
$$J: TX \rightarrow TX, \quad J^2 = -\id.$$
From this, we have the following decompositions 
$$TX_\mathbb{C} = TX \otimes_\mathbb{R} \mathbb{C} = TX^{(1, 0)} \oplus TX^{(0, 1)}.$$
The summands are the $\pm i$-eigenspaces of $J$ and are the \emph{holomorphic} and \emph{antiholomorphic} parts respectively. Furthermore,
\begin{align*} 
TX^*_\mathbb{C}  &= TX^{*(1, 0)} \oplus TX^{*(0, 1)}.\\
\Lambda^{p, q}_\mathbb{C}(X)  &= \Lambda^p(TX^{*(1, 0)}) \otimes_\mathbb{C} \Lambda^q(TX^{*(0, 1)}).\\
\Lambda^k_\mathbb{C}(X)  &= \bigoplus_{p + 1 = k} \Lambda^{p, q}_\mathbb{C}(X).
\end{align*}
Let $\mathcal{A}^{p, q}_X$ be the sheaf of $(p, q)$-differential forms and 
$$\mathcal{A}^{k}_\mathbb{C} X = \bigoplus \mathcal{A}^{p, q}_X.$$
We may define the following differential operators, 
\begin{align*}
d&: \mathcal{A}^k_{X, \mathbb{C}} \rightarrow \mathcal{A}^{k+1}_{X, \mathbb{C}} \\
\partial&: \mathcal{A}^{p, q}_{X} \rightarrow \mathcal{A}^{p+1, q}_{X} \\
\overline{\partial}&: \mathcal{A}^{p, q}_{X} \rightarrow \mathcal{A}^{p, q+1}_{X} 
\end{align*}
On a complex manifold, $d = \partial + \overline{\partial}$ (that is, $X$ is integrable) and $d^2 = \partial^2 = \overline{\partial}^2 = 0$. Henceforth, we obtain the \emph{$(p, q)$-Dolbeault cohomology group} 
$$H^{p, q}(X) = H^q(\mathcal{A}^{p, \cdot}(X), \overline{\partial}) = \frac{\ker(\overline{\partial}: \mathcal{A}^{p, q}(X) \rightarrow \mathcal{A}^{p, q+1}(X))}{\Ima( \overline{\partial}: \mathcal{A}^{p, q-1}(X) \rightarrow \mathcal{A}^{p, q}(X))}.$$
\begin{thrm}[Dolbeault]
$H^{p, q}(X) \cong H^q(X, \Omega^p_X)$
\end{thrm}
This is simply the complex-geometric analogue of De Rham's theorem. For further definitions, we require the Hodge-$\ast$ operator, 
$$\ast: \Lambda^k_{\mathbb{C}} X \rightarrow \Lambda^{2n-k}_{\mathbb{C}} X$$
where $2n$ is the real dimension. Implicitly, this definition requires a Hermitian metric $g$ on the complex manifold $X$, which we have thanks to a partition of unity argument. From this, we again obtain another set of differential operators.
\begin{align*}
d^* &= - \ast \circ d \circ *, \quad \Delta = d^*d+dd^* \\
\partial^* &= - \ast \circ \overline{\partial}  \circ *, \quad \Delta_{\partial} = \partial^*\partial+\partial \partial^* \\
\overline{\partial}^* &= - \ast \circ \partial \circ *, \quad \Delta_{\partial} = \overline{\partial}  ^*\overline{\partial}  +\overline{\partial} \overline{\partial} ^*.
\end{align*}
We also have $\mathcal{H}^{p, q}_{\overline{\partial}} (X, g)= \{\alpha \in \mathcal{A}^{p, q}_\mathbb{C}(X, g), \Delta_{\overline{\partial}}\alpha = 0\}$. We can also define $\mathcal{H}^{p, q}_{\overline{\partial}} (X, g)$ analogously. 
\begin{thrm}[Hodge decomposition] With respect to the Hermitian metric $g$ on $X$, 
\begin{align*}
\mathcal{A}^{p, q}(X, g) &= \partial \mathcal{A}^{p-1, q}(X, g) \oplus \mathcal{H}^{p, q}_{\partial} (X, g) \oplus  \partial^* \mathcal{A}^{p, q+1}(X, g) \\
&= \overline{\partial} \mathcal{A}^{p-1, q}(X, g) \oplus \mathcal{H}^{p, q}_{\overline{\partial}} (X, g) \oplus  \overline{\partial}^* \mathcal{A}^{p, q+1}(X, g).
\end{align*}
\end{thrm} 
It has a very important corollary that motivates the study of Higgs bundles. 
\begin{cor} 
$$H^k_{\text{sing}}(X, \mathbb{C}) = \bigoplus_{p+q=k} H^{p, q}(X).$$
\end{cor} 
Thus, Hodge's decomposition is the bridge between the topological and Dolbeault worlds. For the rank 1 case ($k = 1$) in particular, the corollary states that
\begin{align*}
H^1_{\text{sing}}(X, \mathbb{C}) &= H^{1, 0}(X) \oplus H^{0, 1}(X) \\
&= H^1(X, \Omega^0_X) \oplus H^0(X, \Omega^1_X) \\
&=  H^1(X, \mathcal{O}_X) \oplus H^0(X, \Omega^1_X) \quad \text{by definition} \\
&= H^0(X, \Omega^1_X)^* \oplus H^0(X, \Omega^1_X) \quad \text{by Serre Duality} 
\end{align*}
Through an extreme exact sequence arguement (see Eoin's talk!), we have that
\begin{align*}
\Hom(\pi_1(X), \GL(1, \mathbb{C}^*) = H^1(X, \mathbb{C}^*) = \Jac X \oplus H^0(X, \Omega^1_X).
\end{align*}
What Hodge theory is really saying is that
$$\{1- \text{dim representations of } \pi_1(X)\} \leftrightarrow \{(E, \varphi) : E \text{ holomorphic line bundle of degree 0}, \varphi \text{ holomorphic 1-form}\}$$
Non-abelian Hodge theory seeks to generalise this phenomenon:
$$\{n- \text{dim representations of } \pi_1(X)\} \leftrightarrow \{(E, \varphi) : E \text{ holomorphic vector bundle of degree 0}, \varphi \text{ holomorphic $n$-form}\}.$$
The right hand side is merely the Higgs bundle. 
\end{document}
